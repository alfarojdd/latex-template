% ======================================================================
% !TeX program = xelatex

\documentclass{report}

% Estilo UMA principal con Malacitana
\usepackage{template/sty_files/template-uma}

% Portada
\usepackage{template/sty_files/uma-cover}

% Paquetes útiles
\usepackage{graphicx}
\usepackage{float}
\usepackage{hyperref}

\begin{document}

% ======================================================================
% PORTADA
% ======================================================================
\UMACover
    {Título de la Memoria o Práctica}
    {Juan de Dios Alfaro López}
    {Grado en Ingeniería Informática}
    {Asignatura / Facultad / Departamento}
    {\today}

% ----------------------------------------------------------------------
% ======================================================================
% RESUMEN
% ======================================================================
\begin{abstract}
Aquí va un resumen breve del trabajo.
Explica qué se hace, por qué y cómo.
\end{abstract}

\newpage
% ======================================================================
% ÍNDICE
% ======================================================================
\tableofcontents
\clearpage

% ======================================================================
\chapter{Introducción}
% ======================================================================

Aquí puedes empezar a redactar cualquier memoria.

\section{Contexto}
Texto del contexto.

\section{Objetivos}
Lista de objetivos.

% ======================================================================
\chapter{Desarrollo del trabajo}
% ======================================================================

\section{Material empleado}

\begin{figure}[H]
    \centering
    \includegraphics[width=0.6\textwidth]{images/ejemplo.png}
    \caption{Ejemplo de imagen.}
\end{figure}

% ======================================================================
\chapter{Conclusiones}
% ======================================================================

Escribe aquí las conclusiones.

% ======================================================================
% Bibliografía
% ======================================================================
\begin{thebibliography}{99}

\bibitem{ejemplo}
Autor. *Título del recurso*. Año.

\end{thebibliography}

\end{document}
